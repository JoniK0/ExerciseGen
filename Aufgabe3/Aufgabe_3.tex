\documentclass{article}
\usepackage{pgfplots}
\usepackage{tikz}
\usepackage{amsmath}
\usepackage{etoolbox}
\AtBeginEnvironment{align}{\setcounter{equation}{0}}
\allowdisplaybreaks
\pgfplotsset{compat=1.3}

\usepgfplotslibrary{fillbetween}
\usetikzlibrary{arrows.meta}
\pgfplotsset{compat=1.5.1}


\title{Durandi}
\normalsize
\author{Erjon Konjuhi}
\date{\today}

\begin{document}
\newcommand{\aufgabe}[4]{

\section{Vektorgeometrie Aufgabe 13}


\pgfmathsetmacro{\m}{#1}
  \pgfmathsetmacro{\q}{#2}
  \pgfmathsetmacro{\y}{#4}
  \pgfmathsetmacro{\z}{#3}
  \pgfmathsetmacro{\result}{\m*\z+\q}



  \noindent\fbox{%
    \parbox{\textwidth}{%
     \textbf{\underline{Aufgabe}:} Zeige rechnerisch, ob der Punkt P(\z,\y) auf dem Graphen der Funktion $y=\m x+\q$ liegt.
    }%
  }
  \vspace{1cm}


  %\pgfmathsetmacro{\m}{1.2}
  %\pgfmathsetmacro{\q}{3}
  %\pgfmathsetmacro{\y}{3}
  %\pgfmathsetmacro{\z}{0}
  %\pgfmathsetmacro{\result}{\m*\z+\q}



  \begin{tikzpicture}
    \centering
    \begin{axis}[
      axis lines=middle,
      axis line style={Stealth-Stealth,very thick},
      xmin=-5.5,xmax=5.5,ymin=-5.5,ymax=5.5,
      xtick distance=1,
      ytick distance=1,
      xlabel=$x$,
      ylabel=$y$,
      title={Lineare Funktion},
      grid=major,
      grid style={thin,densely dotted,black!20},
      width = 1\textwidth,
      height=1\textwidth
      ]
      \addplot [black ,domain=-5:5] {x*\m +\q};
      \filldraw[black] (axis cs:\z,\y) circle(2pt) node[anchor=west]{P(\z,\y)};

    \end{axis}

  \end{tikzpicture}


  \noindent
  \textbf{\underline{Lösung}:} Um festellen zu können, ob ein gegebener Punkt auf einer linearen Funktion liegt, muss die x-Koordinate des Punktes in die Funktion eingesetzt werden. Wenn der y-Wert der gleiche ist wie die y-Koordinate, dann liegt der Punkt auf der Gerade, andernfalls nicht.

  \begin{align}
    y &= mx+q\\
      &= \m x+\q\\
    \y &\stackrel{!}{=} \m \cdot \z + \q\\
    \y &\stackrel{!}{=} \result
  \end{align}

}

%{m}{q}{x}{y}
\aufgabe{1.2}{0}{1}{1.2}
\newpage
\aufgabe{2}{2}{1}{4}
\newpage
\aufgabe{0.3}{-2}{4}{-3}
\newpage
\aufgabe{5}{1}{2}{1}

\end{document}
