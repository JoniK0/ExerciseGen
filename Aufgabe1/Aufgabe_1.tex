\documentclass{article}
\usepackage{pgfplots}
\usepackage{tikz}
\usepackage{amsmath}
\usepackage{etoolbox}
\AtBeginEnvironment{align}{\setcounter{equation}{0}}
\allowdisplaybreaks
\pgfplotsset{compat=1.3}

\usepgfplotslibrary{fillbetween}
\usetikzlibrary{arrows.meta}
\pgfplotsset{compat=1.5.1}
\usepackage{siunitx}
\sisetup{
  round-mode = places,
  round-precision = 3
}





\title{Durandi}
\normalsize
\author{Erjon Konjuhi}
\date{\today}

\begin{document}
\newcommand{\aufgabe}[1]{

  \section{Analysis S.43 Aufgabe 21 mit Parameter A=#1}

  \centering

  \noindent\fbox{%
    \parbox{\textwidth}{%
     \textbf{\underline{Aufgabe}:} Der Graph der Funktion $f:y=(a-x)\sqrt{x},a>0$, schliesst mit der x-Achse ein Flächenstück vom Inhalt A = #1 ein. Welchen Wert hat a?
    }%
  }
  \vspace{1cm}

    \def\A{#1}
    \pgfmathsetmacro{\b}{((15 * \A)/4)^(1/5)}
    \pgfmathsetmacro{\a}{\b^2}

    \begin{tikzpicture}[scale=0.7]

    \begin{axis}[
      axis lines=middle,
      axis line style={Stealth-Stealth, thick},
      xmin=-2,xmax=\a+2,ymin=-2,ymax=\a+2,
      xtick distance=1,
      ytick distance=1,
      xlabel=$x$,
      ylabel=$y$,
      grid=major,
      style={very thick},
      grid style={thin, densely dotted, black!20},
      width=1\textwidth,
      height=1\textwidth
      ]

      \addplot [smooth, black, domain=0:\a+1, samples=1000,name path=A] {(\a-x) * sqrt(x)};

      \path[name path=axis] (axis cs:0,0) -- (axis cs:\a,0);
      \tikzfillbetween [of=A and axis]{cyan!50};


    \end{axis}



  \end{tikzpicture}
\large

\textbf{\underline{Lösung}:} Die Fläche der Funktion kann mithilfe des Integrals berechnet werden. Die untere Grenze muss 0 sein, aufgrund des Definitionsbereichs der Funktion. Die obere Grenze muss immer $a$ sein, da die Funktion dort die $x$-Achse schneidet. \newline

\begin{align}
  \A &\stackrel{!}{=} \int_{0}^{a} (a-x)\sqrt{x} \,dx \\
  &= \int_{0}^{a} ax^{1/2}-x^{3/2}\,dx \\
  &= [\frac{2}{3} ax^{3/2}-\frac{2}{5} x^{5/2}]_{0}^{a} \\
  &= \frac{2}{3}a^{5/2}-\frac{2}{5}a^{5/2} \\
  a^{5/2}(\frac{2}{3}-\frac{2}{5}) &= \A \\
  a^{5/2} &= \frac{15 \cdot \A}{4} \\
  \sqrt{a}^{5} &= \frac{15 \cdot \A}{4} \\
  a^{2}\sqrt{a} &= \frac{15}{4}\A \\
  a^{5} &= (\frac{15}{4}\A)^{2} \\
\end{align}
  Durch vereinfachen kommt man auf die allgemeine Lösung $\sqrt[5]{\frac{15}{4}a}$, in der man die Fläche $A$ einsetzen kann und somit $a$ ausrechnet.
\begin{align}
  a &= \sqrt[5]{\frac{15}{4}\A}\\
  a &= \num[round-pad = false]{\a}
\end{align}

}

\aufgabe{90}
\newpage
\aufgabe{30}
\newpage
\aufgabe{50}
\newpage
\aufgabe{10}

\begin{figure}[h!]
  \centering
  \includegraphics[scale=0.3]{meme}

\end{figure}

\end{document}
